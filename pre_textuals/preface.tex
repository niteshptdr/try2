% Prefácio
\chapter*[Preface]{Preface} \index{Preface}

\newepigraph{From my part I know nothing with any certainty, but the sight of the stars makes me dream.}
{Vincent Van Gogh}

When Prof. Fred suggested me to study Fuzzy Time Series - I need to confess - I became excited. Because one of the most fascinating issues on scientific research is to deal with uncertainty. Uncertainty is pervasive, omnipresent and self propagated. Pliny the Elder, early on first century of Cristian Age, stated that ``the only certainty is the uncertainty''. The mankind expanded the boundaries of the knowledge and some uncertainties could be reduced or eliminated. Others, however, remain irreducible. And here we are!

I always felt uncomfortable with the mechanistic and deterministic view of the world. The advances of science have forced us to assume some limitations of our knowledge and accept the separation between our known-knowns, the known-unknowns and even of the unknown-unknowns. We know now that we live in a fuzzy and probabilistic world. 

And until here we just talked about the present and the past. Things get even more interesting when we try to look ahead and predict the future. If we can't measure accurately some natural, social and economical processes, due to instrumentation limitations for example, and these processes are also  intrinsically non-deterministic, these uncertainties combined make the forecasting task complex and barely precise.

The fog of uncertainty becomes yet more dense as the forecasting horizon goes away: the forecasting methods need to into take account all uncertainties on present to forecast ranges of possibilities on future. When we look more than one step in the future the forecasting method should consider all possible combinations in the range of variation of each past step - and this increases the complexity and the output uncertainty.

With this research problem in hand, many ideas in mind, and a lot of excitement, we expect to give some contributions to this field. We focused on non-deterministic processes and assume that all measurements are not completely accurate, every single value actually represents a fuzzy neighborhood. We propose to bring the fuzzy time series to the domain of probabilistic forecasting. 

I hope you enjoy this work as I enjoyed dreaming and implementing it. 
